\section{Программные средства моделирования и
  автоматизированного проектирования конструкций РЭС}

% KiCAD

Из-за моей привержинности к
свободному программному обеспечению ~\cite{GNU-philosophy},
% GNU-philosophy
% дату доступа выбрать по дневнику
% https://www.gnu.org/philosophy/free-sw.html
мною была выбрана САПР
с откртым исходным кодом  \textit{KiCAD} ~\cite{kicad-license}.
% kicad-license
% https://www.kicad.org/about/licenses/
Данная САПР, на настоящий момент,
является одним из самых передовых решений,
среди всех САПР с открытм исходным.
Его разработку также спонсировал \textit{CERN} ~\cite{kicad-sponsors}.
% kicad-sponsors
% https://www.kicad.org/sponsors/inkind/

В данной САПР было осуществлено создание
актуальной принципиальной cхемы устройства,
а также трассировка печатной платы, на основе связей между элементами,
импортированных из схемотехнической
части программы \textit{kicad-schdoc} ~\cite{kicad-doc-schdoc}.
% kicad-doc-schdoc
% link to some of the
% SimulIDE, KTechLab Proteus

Во время разработки принципиальной схемы печататной платы,
была допущена ошибка, из-за котрой транзистор,
размещённый на принципиальной схеме был подключен не теми выводами,
которые были обозначены на изначальной модернизируемой
приципиальной схеме.
О своей оплошности я был осведомлён уже в тот момент,
когда выслал файлы по почте в отдел производство.
Я был проинформирован о недочёте в схеме начальником производственного отдела.
Это дало мне ценный урок:
при любых сомнениях в схемотехнической части разрабатываемого устройства,
следует использовать САПР,
с возможностью симуляции сложных логических схем или микроконтроллеров.
Среди изученных мною ранее
таковой является \textit{Proteus} ~\cite{Proteus-About}.
% Proteus-About
% Proteus site
Её аналог в мире
свободного программного обеспечения — \textit{SimulIDE} ~\cite{SimulIDE—About}.
%

Для разработки данной печатной платы было достаочно
использоваать САПР ПП \textit{KiCAD},
однако стоит отметить, что использование специальных САПР симуляторов,
например \textit{Proteus} или \textit{KiCAD} существенно бы ускорило,
процесс изготовления и введения устройства в эксплуатацию
за счёт экономии времени на этапе исправления ошибок в принципиальной схеме.

\newpage

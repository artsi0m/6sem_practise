\section{Этапы проектирования конструкции РЭС и их характеристика}

Во время работы в малом предприятии трудно заметить
четкие границы перехода между этапами конструирования РЭС,
так как процесс конструрования, кроме того, что увлекателен сам по себе,
в данном случае ещё и не усложнён бюрократической волокитой,
а также нуждой согласовывать каждый этап.
Хотя это можно также списать на тот факт,
что во время прохождения практики я не был зачислен в основной штат работников и
потому по отношению ко мне действовали некоторой степени послабления.


Однако остаётся, по крайней мере, два способа
провести разделение процесса проектирования
на этапы:

\begin{enumerate}
\item По стадиям разработки конструкторской документации
\item По стадиям работы в САПР
\end{enumerate}

Обычно, в техническо литературе, приводится пять стадий разработки консрукторсокй документации ~\cite{Rotkop1976}.

Эти стадии включают в себя:

\begin{enumerate}
\item Техническое задание         
\item Техническое предложение     
\item Эскизный проект             
\item Технический проект
\item Выпуск рабочей документации
\end{enumerate}


Стадии работы в САПР, я бы описал следующим образом:
\begin{enumerate}
\item Создание библиотеки схемных элементов и библиотеки посадочных мест.
\item Созадени принципиальной схемы со связями между элементами.
\item Верификация принципиальной схемы
  и экспорт связей в часть САПР, предназначенную для трассировки
   «\textit{ERC}».
  % Electric Rule Checker.
\item Трассивровка печатной платы и верификация соблюдения
  установленных правил трассировки
  «\textit{DRC}».
  % Design Rule Checker
\end{enumerate}

Первый этап работы в САПР мною был пропущен,
по той причине, что kicad обладает большой встроенной
библиотекой посадочных мест~\cite{KiCAD-included-footprints}.

Точно также мною была найдена готовая
схемная библиотека соотвествующая ГОСТ ~\cite{KiCAD-symbols-gost}
% https://github.com/KiCad-RU/kicad-symbols-gost

Упорно работая я пришел к результату,
когда верефикация принципиальной схемы «\textit{ERC}»
не показывала предупреждений о наружении.

Спустя несколько дней к такому же результату,
я пришел выполняя верефикацию разведённой печатной платы «\textit{DRC}».

Однако прохождение верефикации не уберегло,
меня от схемотехнической ошибки,
с неправильно подключенным транзистором.
Тут могло помочь только внимание к сложным местам,
потому что именно место, где я далее допустил ошибку,
изначально показалось мне самым сложным, как в создании схемы,
так и в трассировке платы.

\newpage

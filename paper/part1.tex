\section{Структура и деятельность служб предприятия}

Компания ООО «РТЕ Сервис» работает с 2009 года
и предлагает широкий диапазон услуг и оборудования, используемого, для
построения автоматизированных систем
комплексного учета потребления энергоресурсов
самой разной степени сложности ~\cite{rte-about}.

На сегодняшний день компания предлагает
большой перечень оборудования собственного производства
от интеллектуальных приборов учета электроэнергии
до комплексов, созданных для экономии потребления энергоресурсов
и оптимизации затрат предприятия ~\cite{rte-about}.

Приоритетное направление деятельности компании – участие в разработке,
производстве и внедрении комплексных систем учета, удовлетворяющих
запросам отечественного рынка и дающие максимальный эффект от их
эксплуатации ~\cite{rte-about}.

Благодаря техническим и программным элементам ИВК «Энергобаланс»,
созданного специалистами компании,
обеспечивается построение\\
автоматизированных систем контроля
и учета энергоресурсов (АСКУЭ) – от простейших,
с несколькими счетчиками,
до территориально распределенных с сотнями и даже тысячами приборов учета,
которые предназначены для технического
и коммерческого учета энергоресурсов ~\cite{rte-about}.

На момент прохождения мною практики
в нём работало суммарно не более 15 человек.
Из чего следует вывод,
что данная компания относится к классу малых предприятий ~\cite{mal-biznes-ru}.


Данная компания занимается, как производством оборудования,
так и обслуживанием его.

Из этого факта можно провести структурное разделение компании
по следующему принципу:
\begin{itemize}
\item Отдел кадров.
\item Отдел сервиса.
\item Отдел производства.
\end{itemize}

Географически же, предприятие располагалось таким образом,
что отдел кадров располагался в Минске. Отдел сервиса,
производящий обслуживание оборудования,
находился с отделом кадров в одном помещении, а отдел
производства был расположен в городе Молодечно.

Таким образом вышло,
что я проходил практику под начальством не кого-либо из отдела производства,
а при отделе сервиса.

Отдел сервиса, в котором я проходил практику, кроме того,
что отвечал на звонки пользователей УСПД и работал с
ИВК «Энергобаланс» был занят отладкой отказавшего
оборудования.

Зачастую это было и оборудование изначально произведённое
какой-то другой компанией.

\newpage

% Local Variables:
% compile-command: "sh build.sh"
% End:

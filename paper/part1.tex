\section{Структура и деятельность служб предприятия}

Копнения ООО «РТЕ Сервис» предоставляет оборудование и предлагает услуги
в области автоматизированных систем коммерческого учёта энергии.

На момент прохождения мною практики в нём работало суммарно не более 15 человек.
Из чего следует вывод,
что данная компания относится к классу малых предприятий ~\cite{mal-biznes-ru}.
% mal-biznes-ru
% [2024-09-03]
% https://economy.gov.by/ru/mal-biznes-ru

Данная компания занимается, как производством оборудования,
так и обслуживанием его.

Из этого факта можно провести структурное разделение компании
по следующему приципу:
\begin{itemize}
\item Отдел кадров.
\item Отдел сервиса.
\item Отдел производства.
\end{itemize}

Географически же, предприятие располагалось таким образом,
что отдел кадров располгалася в Минске, так же как и отдел сервиса,
производящий обслуживание оборудования.

Таким образом вышло,
что я проходил практику не под начальством кого-либо из отдела производства,
а под кураторством сотрудников отдела сервиса.

Однако, я тоже сыграл свою роль в таком решение о выборе технического задания.
Дело в том, что компания ООО «РТЕ Сервис» занимается производством
критического оборудования, для которого высока цена ошибки.
Оно включает в себя счётчики электроэнергии,
предоставляющие данные о потреблении на коммуникаторы и УСПД,
% устройство сбора и передачи данных
которые, в свою очередь, отправлют эти данные на ИВК.
% измерительно вычислительный комплекс

Отдел сервиса, в котором я проходил практику, кроме того,
что отвечал на звонки пользователей УСПД и работал с
ИВК «Энергобаланс» был занят отладкой отказавшего
оборудования, которое находилось в обслуживании отдела сервиса.
Зачастую это было и оборудование изначально произведённое
какой-то другой компанией.

\newpage

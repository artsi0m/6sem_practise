\section{Проектная документация для РЭС}

\subsection{Перечень и характеристика основных конструкторских документов}

Поскольку, данное радиоэлектронное средство,
используется в условиях работы в закрытом, проветриваемом помещении,
можно считать, что условия работы описаны в ГОСТ-УХЛ4.2.

Исходя из этого стандарта и технического задания,
было решено выбрать вариант исполнения устройства,
при котором как таковой отсуствует корпус.

Опираясь на стандарты ГОСТ ~\cite{GOST-spec}
% gost spec
и ГОСТ ~\cite{GOST-SEL} можно прийти к выводу,
% gost SEL
что заполнение спецификации в данном проекте РЭС не требуется,
по той причине, что устройство является отдельной печатной платой,
а не составной сборочной единицей,
состоящей из корпуса, печатной платы,
отдельных крепежных изделий и жгутовых соединений.

Ровно по этой же причине не был создан сборочный чертёж .

\subsection{Основные правила выполнения схем}

Основное правило выполнения схемы, согласно ГОСТ:
Сигнал идёт по схеме слева направо ~\cite{GOST-schematic-main}.
% GOST-schematic-main — тот ГОСТ, где сказано, что сигнал идёт слева направо.
И согласно этому правилу разъём питания был размещён с левой стороны схемы.

При разработке приципиальной схемы также важна консистентность,
так, на схеме не могут быть выставлены номиналы только у части
всех представленных элементов.
Обозначения номиналов должно либо отсуствовать у каждого элемента,
либо присуствовать у каждого, при условии,
что это не мешает читаетмости принципиальной схемы.
При этом,
единицы измерения номиналов не пишутся на схеме,
пишутся только обозначения коэффициентов.
Так резистор сопротивлением 47 Ом на схеме обозначается как «47», а
резистор сопротивлением 1 кОм, на схеме обозначается как «1к».

\subsection{Правила выполнения
  перечня элементов электрической принципиальной схемы}

Cогласно ГОСТ ~\cite{GOST-SEL} перечень
элементов был выполнен следующим образом:
Элементы в таблице составлены,
согласно своим обозначениям на схеме, в порядке английского алфавита,
а также с учётом увелечения номинала.
При этом была использована опция стандарта,
разрешающая вынесение резисторов и конденсаторов
в начало схемы, как элементов представленных
в самом большом количестве на схеме.

\subsection{Основные правила написания спеицификации на сборочную единицу}

Принимая во внимание,
то что для данной РЭС не было необходимо создание спецификации,
можно, тем не менеее добавить,
что основное различие в заполнение спецификации заключается в том,
что, перечень составляется так, что устройства располагаются в таблице,
согласно обозначениям элементов на схеме,
расположенных по порядку латинского алфавита и номиналу,
в порядке возрастания ~\cite{GOST-spec}.

Спецификация же составляется таким образом,
что в ней элементы расположены согласно своему названию по порядку
кириллического элемента. При этом, при описании элемента,
сначала выписывается имя существительное, а затем имя прилагательное.
Пример: «Резистор Подстроечный» или «Резистор Переменный».

Также отличается сам формат таблицы спецификации.

\newpage

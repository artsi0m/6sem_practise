\section{Проектная документация для РЭС}

\subsection{Перечень и характеристика\\
  основных конструкторских документов}

Поскольку,
данное радиоэлектронное средство,
используется в условиях работы в закрытом, проветриваемом помещении,
можно считать, что условия работы описаны в ГОСТ-15150-69,
разделе УХЛ 4.2 ~\cite{GOST-15150-69}.

Исходя из этого стандарта и технического задания,
было решено выбрать вариант исполнения устройства,
при котором как таковой отсутствует корпус.

Опираясь на стандарты ГОСТ 2.108-68 ~\cite{GOST-spec}
% gost spec
и ГОСТ 2.702-2011 ~\cite{GOST-2-702-2011} можно прийти к выводу,
% gost SEL
что заполнение спецификации в данном проекте РЭС не требуется,
по той причине, что устройство является отдельной печатной платой,
а не составной сборочной единицей,
состоящей из корпуса, печатной платы,
отдельных крепежных изделий.

Ровно по этой же причине не был создан сборочный чертёж .

Такое строение было обусловлено техническим заданием на РЭС.

Также можно добавить, что при выполнении конструкторской документации,
были, в обязательном порядке, использованы следующие ГОСТ:
\begin{enumerate}
\item ГОСТ 2.702 «Электрические схемы»  
\item ГОСТ 2.701 «Виды»                 
\item ГОСТ 2.104 «Основные надписи»     
\item ГОСТ 2.743 «УГО Цифровых схем»    
\item ГОСТ 2.759 «УГО Аналоговых схем»  
\end{enumerate}

Однако стоит отметить, что на предприятии никто не ожидал
строгого следования ГОСТ, касающегося непосредственно
принципиальной схемы. В приорете было скорее правильное изготовление печатной платы.

Основываясь на этом,
я уделял более пристальеное стандартам регламентирующим
выполнение печатной платы, таким как некоторые стандарты IPC,
а точнее тем ограничениям, которые требовалось установить в
автотрассировщике на минимальную ширину дорожки и зазор между ними.

\subsection{Основные правила выполнения схем}

Основное правило выполнения схемы, согласно ГОСТ:
Сигнал идёт по схеме слева направо ~\cite{GOST-2-702-2011}.
% GOST-schematic-main — тот ГОСТ, где сказано, что сигнал идёт слева направо.
И согласно этому правилу разъём питания был размещён с левой стороны схемы.
Это правило также влияет,
на работу в САПР так,
что на этапе создания принципиальной схемы,
автоматическая аннотация элементов схемы осуществляется,
слева-направо, сверху-вниз.
Это режим аннотации,
выбирается в настройках САПР.
% CЮДА можно вставить скриншот с окном из KiCAD.

При разработке принципиальной схемы также важна согласованность,
так, на схеме не могут быть выставлены номиналы только у части
всех представленных элементов.
Обозначения номиналов должно либо отсутствовать у каждого элемента,
либо присутствовать у каждого, при условии,
что это не мешает читаемости принципиальной схемы.
При этом,
единицы измерения номиналов не пишутся на схеме,
пишутся только обозначения коэффициентов.
Так резистор сопротивлением 47 Ом на схеме обозначается как «47», а
резистор сопротивлением 1 кОм, на схеме обозначается как «1к».

\subsection{Правила выполнения
  перечня элементов\\
  электрической принципиальной схемы}

Согласно ГОСТ ~\cite{GOST-2-702-2011} перечень
элементов был выполнен следующим образом:
Элементы в таблице составлены,
согласно своим обозначениям на схеме, в порядке английского алфавита,
а также с учётом увеличения номинала.
При этом была использована опция стандарта,
разрешающая вынесение резисторов и конденсаторов
в начало схемы, как элементов представленных
в самом большом количестве на схеме.

\subsection{Основные правила написания\\
  спецификации на сборочную единицу}

Принимая во внимание,
то что для данной РЭС не было необходимо создание спецификации,
можно, тем не менее добавить,
что основное различие в заполнение спецификации заключается в том,
что, перечень составляется так, что устройства располагаются в таблице,
согласно обозначениям элементов на схеме, расположенных по порядку латинского алфавита и номиналу,
в порядке возрастания ~\cite{GOST-spec}.

Спецификация же составляется таким образом,
что в ней элементы расположены согласно своему названию по порядку
кириллического элемента. При этом, при описании элемента,
сначала выписывается имя существительное, а затем имя прилагательное.
Пример: «Резистор Подстроечный» или «Резистор Переменный».

Также отличается сам формат таблицы спецификации.

Спецификация в общем случае состоит из разделов,
которые располагают в следующей последовательности ~\cite{GOST-spec}:
\begin{enumerate}
\item Документация;
\item Комплексы;
\item Сборочные единицы;
\item Детали;
\item Стандартные изделия;
\item Прочие изделия;
\item Материалы;
\item Комплекты.
\end{enumerate}

Использование спецификации в конкретном данном случае не обязательно,
так как вписать в пункт сборочные единицы
что-либо кроме строчки с непосредственно печатной платой
и блоком питания на 5 Вольт,
необходимом для подключения изделия к сети, не представляется возможным.

По этой причине, данная задача по созданию конструкторской документации,
полностью покрыта созданием перечня элементов.

\newpage

% Local Variables:
% compile-command: "sh build.sh"
% End:

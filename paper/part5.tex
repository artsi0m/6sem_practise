\section{Проектные решения конструкции РЭС}

\subsection{Обоснование актуальности разработки электронного модуля}

В ходе совещания с работниками отдела,
мною был предложен вариант,
когда, основываясь на требованиях сотрудников данного
отдела, я, создвая конструкторскую документацию,
на производства РЭС, ошибки в изготовлении которого не будут
слишком сильно экономически на благополочие фирмы,
но при этом факт успешного изготовления
единичного экземпляра или их пары,
существенно упростил бы работу одного из отделов.

Писать код для ИВК или модернизировать коммуникаторы,
а уж тем более счётчики мне бы не доверили по причине
малого опыта. Поэтому, ко мне пошли на встречу, когда
я предложил выполнить что-то более простое,
но тем не мнее полезное отделу сервиса,
в котором я был занят прохождением практики.


Разработка данного радиоэлектронного средства обосновывалась
насущной необходимостью работников предприятия,
которым требовалось устройство упрощающее взаимодействие
между счетчиком и персональным компьютером.

\subsection{Составление технического задания на модернизацию электронного модуля}

Из-за того, 
что я проходил практику не в отделе производства,
а под начальством сотрудников отдела сервиса,
техническое задание на РЭС,
было устно сформулировано работниками сервиса
в устной форме,
исходя из требований этого отдела.

Согласно озвученному техническому заданию,
была составлена схема преобразователя интерфейсов
\textit{RS232 — RS485}.

Модернизация заключалась в том,
что на готовой схеме разъём \textit{DB9}
со стороны включения интерфейса \textit{RS232}
был заменён на клемму с тремя разъёмами.

Такой выбор был сделан на основании
желания сотрудников раздела
упростить подключение счётчика с интерфейсом \textit{RS485}
к компьютеру, c наличествующим интерфейсом \textit{RS232},
но не снабженного при том разъёмом \textit{DB9}.

Более того: разъём \textit{DB9}
содержит слишком избыточное количество
пинов для подключения по интерфейсу \textit{RS232}.

Конечно же, существуют разные версии стандарта этого интерфейса,
но в минимальной и достаточной комлектации,
необходимы всего три пина:
\begin{itemize}
\item RX
\item TX
\item GND
\end{itemize}

Сам по себе данный стандарт является уточнением
стандарта \textit{UART} и часто,
в литературе может быть описано, как
«последовательный порт компьютера».

Недостаток данного интерфейса, в том,
что в отличие от \textit{RS485}
он не предназначен для передачи данных на дальнее расстояние.
По этой причине на многих старых счетчиках,
изготовленных вне ООО «РТЕ Сервис» был
использован именно этот интерфейс,
как часто вариант размещения,
когда коммуникатор был размещён в одному шкафу с счётчиком,
был недоступен.
В такой ситуации требовалась передача данных на более дальнее ростояние.
И это именно тот случай, когда интерфейс
основанный на приципе работы витой пары,
подходит как никак лучше.


\subsection{Пояснение схемы электрической принципиальной электронного модуля}

\subsection{Выбор элементов и комплектующих изделий для электронного модуля входящего в состав РЭС}

\subsection{Конструкторское проектирование электронного модуля}

\subsection{Компьютерные программы используемые для конструкторского проектирования}

\newpage
